%Dear Students,
%
%The pProject synopsis can be any format. It should be handed in as an Assignment on DTU Inside Wednesday Nov 1st. Hand in one for each synopsis for each group.
%
%It must must contain:
%
%1. Title
%2. Name and study id of all group members
%3. Background and motivation
%4. Milestones in 3 to 5 bulletpoints
%5. References.
%
%it should not be more than 1 page long. The main requirement is that the project scope is realistic. Being ambitious is good but if you for example are both implementing your own models and using a new data set then there should be a more safe plan b using a standard data set and/or model.
%
%Something else: Remember that the TAs and me will still come to B306 aud 33 every Monday from 13. (The incident today was an exception.)
%
%Best regards,
%
%Ole

\documentclass[utf-8]{article}
\usepackage[utf8]{inputenc}
\usepackage{amsfonts}
\usepackage{amssymb}
\usepackage{amsmath}
\usepackage{graphicx} 
\usepackage[pdftex=true,colorlinks=true,linkcolor=blue,linktocpage=true]{hyperref} 
\usepackage{cleveref} % has to be loaded after hyperref
%\input{tp_mathgeneral.tex}
%\input{tp_nc_vector.tex}
%\input{tp_codigo.tex}
\newcommand{\urb}{\textit{UrbanSound8K}}
\begin{document}
\section*{Synopsis for 02456 Project "Various Deep Learning Architectures for Urban Sound Classification"}
\begin{tabular}{ll|ll}
s161041 &Sébastien Demortain &s161027 &Péter Semság\\
 s161174 &Lorenzo Belgrano &s161463 &Benjamin Jüttner
\end{tabular}

\subsection*{Background and Motivation:}
Sound classification is a task commonly solved by RNNs rather than CNNs, which in turn are rather suitable for image data. However, since a spectrogram of an audio sequence can be interpreted as an image, CNNs too can be used for sound data, as was done e.g. in \cite{pizza}. Therefore, sound data is a good opportunity to work with two of the architectures we learned in 02456, namely CNNs and RNNs. The dataset chosen for the project was the \urb\cite{urban}, which is a collection of over 8000, up to 4 seconds long, audios from urban environments with labels such as dog barking or jackhammer.  

\subsection*{Milestones:} %should probably not be too detailed
\begin{enumerate}
\item (also safe plan B) Reproduce the CNN architecture proposed in \cite{pizza}, with each audioclip processed into several $60\times41$ pixel spectrograms.\label{purepizza}
\item Same architecture and same data as in Milestone \ref{purepizza}, but now train easier observations first and more difficult observations afterwards (see \textit{curriculum learning} \cite{curr}). See if the performance improves.
\item Implement an architecture combined of CNN and RNN (maybe realizing some of the ideas in \cite{deepspeech}).\label{cnnrnn} %if the network architecture allows, use sequences of different lenghts here? And do the curriculum stuff?
\item %Take good network architecture found in Milestones \ref{purepizza} through \ref{cnnrnn} and e
Experiment with mixed data:
(i) Artificially overlap two audios of classes $a$ and $b$ and make it one observation. The network should give a softmax output where the highest values are in $a$ and $b$.  
(ii) Concatenate audios, e.g. dog -- jackhammer -- silence -- jackhammer. Use CTC\cite{ctc, ctctf} to  automatically segment the new audio and give a label for each segment.  
%Test the networks we have trained, optimzied and evaluted, on mixture audioclips, and see which classes have the strongest representation in the softmax output of the network.
%(will probably involve CTC)
\end{enumerate}


\begin{thebibliography}{5}
\bibitem{pizza} K. J. Piczak: "ENVIRONMENTAL SOUND CLASSIFICATION
WITH CONVOLUTIONAL NEURAL NETWORKS", in \textit{2015 IEEE INTERNATIONAL WORKSHOP ON MACHINE LEARNING FOR SIGNAL PROCESSING}, Sept. 17–20, 2015, Boston, USA
\bibitem{urban} J. Salamon, C. Jacoby, and J. P. Bello, "A dataset and
taxonomy for urban sound research", in \textit{Proceedings
of the ACM International Conference on Multimedia.}
ACM, 2014, pp. 1041–1044.
\bibitem{curr} Y. Bengio, J. Louradour, R. Collobert, J. Weston: "Curriculum Learning"
\bibitem{deepspeech} Baidu Research – Silicon Valley AI Lab: "Deep Speech 2: End-to-End Speech Recognition in
English and Mandarin"
\bibitem{ctc} A. Graves, S. Fernández, F. Gomez, and J. Schmidhuber: "Connectionist temporal classification: Labelling
unsegmented sequence data with recurrent neural networks." In \textit{ICML}, pages 369–376. ACM, 2006.
\bibitem{ctctf} TensorFlow implementation of \cite{ctc}:  
\verb@https://www.tensorflow.org/versions/r0.12/@\\
\verb@api_docs/python/nn/connectionist_temporal_classification__ctc_@ 
\end{thebibliography}

\end{document} 